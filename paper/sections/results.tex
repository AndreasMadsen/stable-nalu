\section{Experimental results}
\label{sec:experiments}

\subsection{Arithmetic datasets}

The arithmetic dataset is a replica of the "simple function task" shown in \cite{trask-nalu}.
The goal is to sum two subsets of a vector and perform a arithmetic operation as defined below
\begin{equation}
t = \sum_{i = a_{\mathrm{start}}}^{a_{\mathrm{end}}} \mathbf{x}_i \circ \sum_{i = b_{\mathrm{start}}}^{b_{\mathrm{end}}} \mathbf{x}_i \quad \text{where } \mathbf{x} \in \mathbb{R}^n, x_i \sim \mathrm{Uniform}[r_{\mathrm{lower}}, r_{\mathrm{upper}}], \circ \in \{+, -, \times\}
\label{eq:arithmetic-problem}
\end{equation}
where $n$, $r_{\mathrm{lower}}$, $r_{\mathrm{upper}}$, $\circ$, the subset size and subset overlap are dataset parameters that we use to test the models ability to learn.
We define a set of default parameters, see table (table \ref{tab:simple-function-task-defaults}).
When probing a specific dataset parameter, e.g. subset overlap, the default will be the used for the remaining parameters.

\begin{table}[H]
\caption{Default dataset parameters}
\label{tab:simple-function-task-defaults}
\centering
\begin{tabular}{r l}
\toprule
 Parameter name & Default value \\
 \midrule
 Input size & 100 \\
 Subset ratio & 0.25 \\
 Overlap ratio & 0.5 \\
 Interpolation range & $U[1,2]$ \\
 Extrapolation range & $U[2,6]$ \\
 \bottomrule
\end{tabular}
\end{table}

\subsubsection{Criterion}

The goal is to achieve a solution that is acceptably close to a perfect solution. To evaluate if a model instance solves the task, the MSE is compared to a known nearly-perfect solution on the extrapolation range.

If $\mathbf{W}_1, \mathbf{W}_2$ defines the weights of the fitted model, and $\mathbf{W}_1^\epsilon$ is nearly-perfect and $\mathbf{W}_2^*$ is perfect (example in equation \ref{eq:nearly-perfect-solution-example}), the success criteria is $\mathcal{L}_{\mathbf{W}_1, \mathbf{W}_2} < \mathcal{L}_{\mathbf{W}_1^\epsilon, \mathbf{W}_2^*}$, measured on the extrapolation error. Meaning the MSE for the fitted model, should be less than the MSE for a nearly perfect solution.

\begin{equation}
    \mathbf{W}_1^\epsilon = \begin{bmatrix}
    1 - \epsilon & 1 - \epsilon & 0 + \epsilon & 0 + \epsilon \\
    1 - \epsilon & 1 - \epsilon & 1 - \epsilon & 1 - \epsilon
    \end{bmatrix}, \mathbf{W}_2^* = \begin{bmatrix}
    1 & 1
    \end{bmatrix}
    \label{eq:nearly-perfect-solution-example}
\end{equation}

All experiments are evaluated multiple times with different seeds. We define the success rate as the percentage of experiments that achieves success.

A sparsity error is also reported, the is defined in equation \ref{eq:sparsity-error}. This is only considered for model instances that did solve the task.
\begin{equation}
E_\mathrm{sparsity} = \max_{h_{\ell-1}, h_{\ell}} \min(|W_{h_{\ell-1},h_\ell}|, |1 - |W_{h_{\ell-1},h_\ell}||)
\label{eq:sparsity-error}
\end{equation}

The first iteration for which $\mathcal{L}_{\mathbf{W}_1, \mathbf{W}_2} < \mathcal{L}_{\mathbf{W}_1^\epsilon, \mathbf{W}_2^*}$, is also reported. Again, only model instances that did solve the task are considered.

For sparsity error and ``solved at'' the 95\% confidence interval is reported.

\subsubsection{Model setup}

To solve the task, we compare the models defined in table \ref{tab:model-defintions}. All models have by default two hidden units in the multiplication layer.

\begin{table}[H]
\caption{Model definitions}
\label{tab:model-defintions}
\centering
\begin{tabular}{r l l}
\toprule
 Model & Layer 1 & Layer 2 \\
 \midrule
 NMU & NAU & NMU \\
 NAU & NAU & NAU \\
 $\mathrm{NAC}_{\bullet}$ & $\mathrm{NAC}_{+}$ & $\mathrm{NAC}_{\bullet}$ \\
 $\mathrm{NAC}_{+}$ & $\mathrm{NAC}_{+}$ & $\mathrm{NAC}_{+}$ \\
 NALU & NALU & NALU \\
 Linear & Linear & Linear \\
 \bottomrule
\end{tabular}
\end{table}

For all experiments $\lambda_{\mathrm{oob}} = 1$ and $\lambda_{\mathrm{bias}} = 0.1 \cdot (1 - \exp(-10^5 \cdot t))$. Gradually scaling the bias regularizer $\mathcal{R}_{\ell,\mathrm{bias}}$ is to ensure it does not interfere with early training. We show the effect of regularization in appendix \ref{sec:appendix:simple-function-task:regualization}.

For all experiments Adam optimization \cite{adam-optimization} with default parameters is used and computed on an HPC cluster using \text{8-Core Intel Xeon E5-2665 2.4GHz} CPUs.

The training dataset is continuously sampled from the interpolation range, a different seed is used for each experiment. Training is done with a mini-batch size of 128 observations.

A fixed validation dataset with $10000$ observations is sampled from the interpolation range. A fixed test dataset with $10000$ observations is sample from the extrapolation range.

Validation error, test error and sparsity error is sampled every $1000$ iterations. To avoid noise from exploration, the best fit in terms of the validation error among the last $100$ samples is used.

\subsubsection{Very simple function}

To empirically validate the theoretical challenges with $\mathrm{NAC}_{\bullet}$ consider the very simple problem shown earlier in figure \ref{fig:nac-mul-eps-issue}. That is, $t = (x_1 + x_2) \circ (x_1 + x_2 + x_3 + x_4)$ for $x \in \mathbb{R}^4$.

Each experiment is conducted 100 times with different seeds, and stopped after 200000 iterations.

The results, in table \ref{tab:very-simple-function-results}, show that NMU has a higher success rate and converges faster. When inspecting the $6\%$ that did not converge, we found the issue to underflow when $w = 0$ in the NMU layer.\todo{Highlight best results in tables.}

\begin{table}[!h]

\caption{\label{tab:very-simple-function-results}Comparison of the success-rate, when the model converged, and the sparsity error for all weight matrices, with 95\% confidence interval on the $t = (x_1 + x_2) \cdot (x_1 + x_2 + x_3 + x_4)$ task. Each value is a summary of 100 different seeds.}
\centering
\begin{tabular}{crllll}
\toprule
\multicolumn{1}{c}{Op} & \multicolumn{1}{c}{Model} & \multicolumn{1}{c}{Success} & \multicolumn{2}{c}{Solved at} & \multicolumn{1}{c}{Sparsity error} \\
\cmidrule(l{3pt}r{3pt}){1-1} \cmidrule(l{3pt}r{3pt}){2-2} \cmidrule(l{3pt}r{3pt}){3-3} \cmidrule(l{3pt}r{3pt}){4-5} \cmidrule(l{3pt}r{3pt}){6-6}
 &  & Rate & Median & Mean & Mean\\
\midrule
 & $\mathrm{NAC}_{\bullet}$ & $13\% {~}^{+8\%}_{-5\%}$ & $5.5 \cdot 10^{4}$ & $5.9 \cdot 10^{4} {~}^{+7.8 \cdot 10^{3}}_{-6.6 \cdot 10^{3}}$ & $7.5 \cdot 10^{-6} {~}^{+2.0 \cdot 10^{-6}}_{-2.0 \cdot 10^{-6}}$\\

\nopagebreak
 & NALU & $26\% {~}^{+9\%}_{-8\%}$ & $7.0 \cdot 10^{4}$ & $7.8 \cdot 10^{4} {~}^{+6.2 \cdot 10^{3}}_{-8.6 \cdot 10^{3}}$ & $9.2 \cdot 10^{-6} {~}^{+1.7 \cdot 10^{-6}}_{-1.7 \cdot 10^{-6}}$\\

\nopagebreak
\multirow{-3}{*}{\centering\arraybackslash $\bm{\times}$} & NMU & $\mathbf{94\%} {~}^{+3\%}_{-6\%}$ & $\mathbf{1.4 \cdot 10^{4}}$ & $\mathbf{1.4 \cdot 10^{4}} {~}^{+2.2 \cdot 10^{2}}_{-2.1 \cdot 10^{2}}$ & $\mathbf{2.6 \cdot 10^{-8}} {~}^{+6.4 \cdot 10^{-9}}_{-6.4 \cdot 10^{-9}}$\\
\bottomrule
\end{tabular}
\end{table}


\subsubsection{Arithmetic operation comparison}
We compare the models on different arithmetic operation $\circ \in \{+, -, \times\}$ used in equation \ref{eq:arithmetic-problem}, results are seen in table \ref{tab:function-task-static-defaults}, where each experiment is trained for $5 \cdot 10^6$ iterations.

For multiplication, the NMU success more often and converges faster. For addition and subtraction, the NAU model converges faster, given the median, and has a more sparse solution.

% latex table generated in R 3.4.3 by xtable 1.8-3 package
% Fri May 10 17:19:56 2019
\begin{table}[H]
\centering
\caption{Shows the sucess-rate for extrapolation < $\epsilon$, at what global step the model converged at, and the sparse error for all weight matrices.} 
\begin{tabular}{lll |rrr}
  \hline
                      &                            & & \multicolumn{1}{l}{          converged.at} & \multicolumn{1}{l}{          sparse.error} & \multicolumn{1}{l}{          success.rate} \\ 
  operation            & model                      &     & \multicolumn{1}{l}{                      } & \multicolumn{1}{l}{                      } & \multicolumn{1}{l}{                      } \\ 
   \hline
  ${a \cdot b}$        & ${\mathrm{NAC}_\bullet}$ &     & $3371250$             & $8.0 \times 10^{-5}$ & $40\%$               \\ 
                       & NALU                       &     & ---                   &      ---              & $0\%$                   \\ 
                       & NMU                        &     & $1571900$             & $1.6 \times 10^{-4}$ & $100\%$              \\ \hline
  $a - b$              & ${\mathrm{NAC}_{+}}$      &     & $6300$                & $1.1 \times 10^{-1}$ & $100\%$              \\ 
                       & linear                     &     & $3300$                & $7.6 \times 10^{-2}$ & $100\%$              \\ 
                       & NALU                       &     & $1963250$             & $9.1 \times 10^{-2}$ & $40\%$               \\ 
                       & NAU                        &     & $3700$                & $4.3 \times 10^{-4}$ & $100\%$              \\ \hline 
  $a + b$              & ${\mathrm{NAC}_{+}}$      &     & $42900$               & $1.4 \times 10^{-1}$ & $100\%$              \\ 
                       & linear                     &     & $21300$               & $1.9 \times 10^{-1}$ & $100\%$              \\ 
                       & NALU                       &     & $81000$               & $1.9 \times 10^{-1}$ & $10\%$               \\ 
                       & NAU                        &     & $15500$               & $1.1 \times 10^{-4}$ & $100\%$              \\ 
   \hline
\end{tabular}
\end{table}


\subsubsection{Exploration of dataset parameters}
To stress test the NMU in comparison with the  $\mathrm{NAC}_{\bullet}$ and NALU, on the multiplication task, the dataset parameters (table \ref{tab:simple-function-task-defaults}) and the size of the multiplication layer are varied. Each experiment runs for 10 different seeds, the results are visualized in in figure \ref{fig:simple-fnction-static-input-size}, \ref{fig:simple-fnction-static-overlap}, \ref{fig:simple-fnction-static-range}, and \ref{fig:simple-fnction-static-subset}. \todo{Update refs for what we keep}

Our results show that the NMU consistently outperform the $\mathrm{NAC}_{\bullet}$ and the NALU for all parameters.

\begin{figure}[H]
\centering
\includegraphics[width=\linewidth]{results/simple_function_static_input_size.pdf}
\caption{Shows the effect of the input size, on the simple function task problem.}
\label{fig:simple-fnction-static-input-size}
\end{figure}

\begin{figure}[h]
\centering
\includegraphics[width=\linewidth]{results/simple_function_static_overlap.pdf}
\caption{Shows the effect of the overlap ratio, on the simple function task problem.}
\label{fig:simple-fnction-static-overlap}
\end{figure}

\begin{figure}[h]
\centering
\includegraphics[width=\linewidth]{results/simple_function_static_subset.pdf}
\caption{Shows the effect of the subset ratio, on the simple function task problem.}
\label{fig:simple-fnction-static-subset}
\end{figure}

\begin{figure}[h]
\centering
\includegraphics[width=\linewidth]{results/simple_function_static_hidden_size.pdf}
\caption{Shows the effect of the hidden size, on the simple function task problem.}
\label{fig:simple-fnction-static-subset}
\end{figure}

\begin{figure}[h]
\centering
\includegraphics[width=\linewidth]{results/simple_function_static_range.pdf}
\caption{Shows the effect of the interpolation range. For each interpolation range, the following extrapolation ranges are used: ${\mathrm{U}[-2,2] \rightarrow \mathrm{U}[-6,-2] \cup \mathrm{U}[2,6]}$, ${\mathrm{U}[0,1] \rightarrow \mathrm{U}[1,5]}$, ${\mathrm{U}[0.1,0.2] \rightarrow \mathrm{U}[0.2,2]}$, ${\mathrm{U}[1,2] \rightarrow \mathrm{U}[2,6]}$, ${\mathrm{U}[10, 20] \rightarrow \mathrm{U}[20, 40]}$.}
\label{fig:simple-fnction-static-range}
\end{figure}
