\section{Experimental results}
\label{sec:experiments}

\subsection{Arithmetic datasets}
\label{sec:arithmetic-dataset}

The arithmetic dataset is a replica of the "simple function task" as shown in \cite{trask-nalu}.
The goal is to sum two random subsets of a vector and perform a arithmetic operation as defined below
\begin{equation}
t = \sum_{i = s_{1,\mathrm{start}}}^{s_{1,\mathrm{end}}} x_i \circ \sum_{i = s_{2,\mathrm{start}}}^{s_{2,\mathrm{end}}} x_i \quad \text{where } \mathbf{x} \in \mathbb{R}^n, x_i \sim \mathrm{Uniform}[r_{\mathrm{lower}}, r_{\mathrm{upper}}], \circ \in \{+, -, \times\}
\label{eq:arithmetic-problem}
\end{equation}
where $n$ (default $100$), $U[r_{\mathrm{lower}}, r_{\mathrm{upper}}]$ (interpolation default is $U[1,2]$ and extrapolation default is $U[2,6]$), the subset size (default 25\%), and subset overlap (default 50\%) are parameters that we use to assess learning capability (see details in appendix \ref{sec:appendix:simple-function-task:data-generation} and the effect of varying the parameters in appendix \ref{sec:appendix-simple-function-task:dataset-parameter-effect}).

\subsubsection{Model evaluation}
The goal is to achieve a solution that is acceptably close to a perfect solution. To evaluate if a model instance solves the task consistently we compare the MSE to a nearly-perfect solution on the extrapolation range over multiple seeds. If $\mathbf{W}_1, \mathbf{W}_2$ defines the weights of the fitted model, and $\mathbf{W}_1^\epsilon$ is nearly-perfect and $\mathbf{W}_2^*$ is perfect (example in equation \ref{eq:nearly-perfect-solution-example}), the success criteria is $\mathcal{L}_{\mathbf{W}_1, \mathbf{W}_2} < \mathcal{L}_{\mathbf{W}_1^\epsilon, \mathbf{W}_2^*}$, measured on the extrapolation error, for $\epsilon = 10^{-5}$.
\begin{equation}
    \mathbf{W}_1^\epsilon = \begin{bmatrix}
    1 - \epsilon & 1 - \epsilon & 0 + \epsilon & 0 + \epsilon \\
    1 - \epsilon & 1 - \epsilon & 1 - \epsilon & 1 - \epsilon
    \end{bmatrix}, \mathbf{W}_2^* = \begin{bmatrix}
    1 & 1
    \end{bmatrix}
    \label{eq:nearly-perfect-solution-example}
\end{equation}
To measure speed of convergence the first iteration for which $\mathcal{L}_{\mathbf{W}_1, \mathbf{W}_2} < \mathcal{L}_{\mathbf{W}_1^\epsilon, \mathbf{W}_2^*}$ is reported with a 95\% confidence interval. Only models that managed to solve the task are included.

We assume an approximate discrete solution with parameters close to $\{-1, 0, 1\}$ is important for inferring exact arithmetic operations.
To measure the sparsity we introduce a sparsity error (defined in equation \ref{eq:sparsity-error}).
Similar to the convergence metric we only considered model instances that did solve the task and report the 95\% confidence interval.
\begin{equation}
E_\mathrm{sparsity} = \max_{h_{\ell-1}, h_{\ell}} \min(|W_{h_{\ell-1},h_\ell}|, |1 - |W_{h_{\ell-1},h_\ell}||)
\label{eq:sparsity-error}
\end{equation}

We evaluate each metric every $1000$ iterations on the test set that uses the extrapolation range, and choose the best iteration based on the validation dataset that uses the interpolation range.

\subsubsection{Arithmetic operation comparison}
We compare models on different arithmetic operation $\circ \in \{+, -, \times\}$. The multiplication models, NMU and $\mathrm{NAC}_{\bullet}$, have an addition layer first, either NAU or $\mathrm{NAC}_{+}$, followed by a multiplication layer. The addition models are just two layers of the same unit. Finally, the NALU model consists of two NALU layers. See explicit definitions in appendix \ref{sec:appendix:comparison-all-models}.

Each experiment is trained for $5 \cdot 10^6$ iterations (details in appendix \ref{sec:appendix:comparison-all-models}). Results are presented in table \ref{tab:function-task-static-defaults}. For multiplication, the NMU succeeds more often and converges faster. For addition and subtraction, the NAU model converges faster, given the median, and has a sparser solution. A larger comparison is in appendix \ref{sec:appendix:comparison-all-models} and an ablation study is in appendix \ref{sec:appendix:ablation-study}.

% latex table generated in R 3.4.3 by xtable 1.8-3 package
% Fri May 10 17:19:56 2019
\begin{table}[H]
\centering
\caption{Shows the sucess-rate for extrapolation < $\epsilon$, at what global step the model converged at, and the sparse error for all weight matrices.} 
\begin{tabular}{lll |rrr}
  \hline
                      &                            & & \multicolumn{1}{l}{          converged.at} & \multicolumn{1}{l}{          sparse.error} & \multicolumn{1}{l}{          success.rate} \\ 
  operation            & model                      &     & \multicolumn{1}{l}{                      } & \multicolumn{1}{l}{                      } & \multicolumn{1}{l}{                      } \\ 
   \hline
  ${a \cdot b}$        & ${\mathrm{NAC}_\bullet}$ &     & $3371250$             & $8.0 \times 10^{-5}$ & $40\%$               \\ 
                       & NALU                       &     & ---                   &      ---              & $0\%$                   \\ 
                       & NMU                        &     & $1571900$             & $1.6 \times 10^{-4}$ & $100\%$              \\ \hline
  $a - b$              & ${\mathrm{NAC}_{+}}$      &     & $6300$                & $1.1 \times 10^{-1}$ & $100\%$              \\ 
                       & linear                     &     & $3300$                & $7.6 \times 10^{-2}$ & $100\%$              \\ 
                       & NALU                       &     & $1963250$             & $9.1 \times 10^{-2}$ & $40\%$               \\ 
                       & NAU                        &     & $3700$                & $4.3 \times 10^{-4}$ & $100\%$              \\ \hline 
  $a + b$              & ${\mathrm{NAC}_{+}}$      &     & $42900$               & $1.4 \times 10^{-1}$ & $100\%$              \\ 
                       & linear                     &     & $21300$               & $1.9 \times 10^{-1}$ & $100\%$              \\ 
                       & NALU                       &     & $81000$               & $1.9 \times 10^{-1}$ & $10\%$               \\ 
                       & NAU                        &     & $15500$               & $1.1 \times 10^{-4}$ & $100\%$              \\ 
   \hline
\end{tabular}
\end{table}


\subsubsection{Evaluating theoretical claims}

To validate our theoretical claim, that the NMU models works better than $NAC_{\bullet}$ for larger $H_{\ell-1}$, we increase the hidden size of the network, thereby adding redundant units. Redundant units are very common neural networks, where the purpose is to fit an unknown function.

Additionally, the NMU model is unlike the $NAC_{\bullet}$ model also capable of understanding inputs that are both negative and positive. To validate this empirically, the training and validation datasets are sampled for $\mathrm{U}[-2,2]$, and then tested on $\mathrm{U}[-6,-2] \cup \mathrm{U}[2,6]$.

Finally, to validate that division and the lack of bias in $NAC_{\bullet}$ are critical issues but that solving these alone are not enough, two additional models compared with. A variant of $\mathrm{NAC}_{\bullet}$ called $\mathrm{NAC}_{\bullet, \sigma}$, that only supports multiplication, by constraining the weights with $W = \sigma(\hat{W})$. And a variant, called $\mathrm{NAC}_{\bullet, \mathrm{NMU}}$, that uses linear weights and bias regularization, identically to that in NMU model.

Figure \ref{fig:simple-function-static-theoreical-claims-experiment} shows that the NMU can both handle a much larger hidden-size, as well as negative inputs, and that solving the division and bias issues alone improves the success rate, but are no enough when the hidden-size is large, as there is no ideal initialization.

\begin{figure}[h]
\centering
\includegraphics[width=\linewidth,trim={0 1.3cm 0 0},clip]{results/simple_function_static_mul_hidden_size.pdf}
\includegraphics[width=\linewidth,trim={0 0 0 0.809cm},clip]{results/simple_function_static_mul_range.pdf}
\caption{Shows that the NMU can handle a large hidden size, and works when the input contains both positive and negative numbers ($U[-2,-2]$).} 
%\caption{Shows the effect of the dataset parameters. For each interpolation range, the following extrapolation ranges are used: ${\mathrm{U}[-2,2] \rightarrow \mathrm{U}[-6,-2] \cup \mathrm{U}[2,6]}$, ${\mathrm{U}[0,1] \rightarrow \mathrm{U}[1,5]}$, ${\mathrm{U}[0.1,0.2] \rightarrow \mathrm{U}[0.2,2]}$, ${\mathrm{U}[1,2] \rightarrow \mathrm{U}[2,6]}$, ${\mathrm{U}[10, 20] \rightarrow \mathrm{U}[20, 40]}$. The uniform sampling ranges are chosen to test the effect of mean, variance, and sign for optimizing.}
\label{fig:simple-function-static-theoreical-claims-experiment}
\end{figure}

\subsection{Product of sequential MNIST}

To compare how easy it is to backpropergation though the arithmetic layers, the arithmetic layers are applied as a recurrent-unit to a sequence of MNIST digits, where the target is to fit the cumulative product. This task is similar to ``MNIST Counting and Arithmetic Tasks'' in \cite{trask-nalu}\footnote{Also uses the same CNN, \url{https://github.com/pytorch/
examples/tree/master/mnist}.}, but use multiplication rather than addition.. Each model is trained on sequences of length 2, and then tested on sequences of length 20 MNIST digits.

Success of convergence is determined by comparing the MSE of each model, with a baseline model that directly computes the sequential product. If the MSE of each model, is less than the upper 1\% MSE-confidence-interval of the baseline model, then the model is considered successfully converged.

Sparsity and ``solved at iteration step'' is determined as described in experiment \ref{sec:arithmetic-dataset}. The validation set is the last 5000 MNIST digits from the training set, which is used to select the best epoch.

In this experiment we discovered that having an unconstrained ``input-network'' causes the multiplication-units to learn an undesired solution, such as $(0.1 \cdot 81 + 1 - 0.1) = 9$. This solves the problem with a similar success-rate, but not in the intended way. To prevent such solution, we regularize the CNN output with $\mathcal{R}_{\mathrm{z}} = \frac{1}{H_{\ell-1} H_\ell} \sum_{h_\ell}^{H_\ell} \sum_{h_{\ell-1}}^{H_{\ell-1}} (1 - W_{h_{\ell-1},h_\ell}) \cdot (1 - \bar{z}_{h_{\ell-1}})^2$. This regularizer is applied to the NMU and $\mathrm{NAC}_{\bullet,\mathrm{NMU}}$ models. See appendix \ref{sec:appendix:sequential-mnist:ablation} for the results, when this regularizer is not used.

Figure \ref{fig:sequential-mnist-prod-results} shows that the NMU does not hindre learning a more complex neural network. And that it can extrapolate to much longer sequences than what is is trained on.

\begin{figure}[h]
\centering
\includegraphics[width=\linewidth,trim={0 0.5cm 0 0},clip]{results/sequential_mnist_prod_long.pdf}
\caption{Shows the ability of each model to backpropergation and extrapolate to larger sequence lengths.} 
\label{fig:sequential-mnist-prod-results}
\end{figure}
