\section{Gradient derivatives}
\label{sec:appendix:gradient-derivatives}

\subsection{Weight matrix construction}
\label{sec:appendix:gradient-derivatives:weight-matrix-construction}
For clarity the weight matrix construction is defined using scalar notation
\begin{equation}
W_{h_\ell, h_{\ell-1}} = \tanh(\hat{W}_{h_\ell, h_{\ell-1}}) \sigma(\hat{M}_{h_\ell, h_{\ell-1}})
\end{equation}

The of the loss with respect to $\hat{W}_{h_\ell, h_{\ell-1}}$ and $\hat{M}_{h_\ell, h_{\ell-1}}$ is then straight forward to derive.
\begin{equation}
\begin{aligned}
\frac{\partial\mathcal{L}}{\partial \hat{W}_{h_\ell, h_{\ell-1}}} &= \frac{\partial\mathcal{L}}{\partial W_{h_\ell, h_{\ell-1}}} \frac{\partial W_{h_\ell, h_{\ell-1}}}{\partial \hat{W}_{h_\ell, h_{\ell-1}}} \\
&= \frac{\partial\mathcal{L}}{\partial W_{h_\ell, h_{\ell-1}}} (1 - \tanh^2(\hat{W}_{h_\ell, h_{\ell-1}})) \sigma(\hat{M}_{h_\ell, h_{\ell-1}}) \\
\frac{\partial\mathcal{L}}{\partial \hat{M}_{h_\ell, h_{\ell-1}}} &= \frac{\partial\mathcal{L}}{\partial W_{h_\ell, h_{\ell-1}}} \frac{\partial W_{h_\ell, h_{\ell-1}}}{\partial \hat{M}_{h_\ell, h_{\ell-1}}} \\
&= \frac{\partial\mathcal{L}}{\partial W_{h_\ell, h_{\ell-1}}} \tanh(\hat{W}_{h_\ell, h_{\ell-1}}) \sigma(\hat{M}_{h_\ell, h_{\ell-1}}) (1 - \sigma(\hat{M}_{h_\ell, h_{\ell-1}}))
\end{aligned}
\end{equation}

As seen from this result, one only needs to consider $\frac{\partial\mathcal{L}}{\partial W_{h_\ell, h_{\ell-1}}}$ for $\mathrm{NAC}_{+}$ and $\mathrm{NAC}_{\bullet}$, as the gradient with respect to $\hat{W}_{h_\ell, h_{\ell-1}}$ and $\hat{M}_{h_\ell, h_{\ell-1}}$ is just a multiplication on $\frac{\partial\mathcal{L}}{\partial W_{h_\ell, h_{\ell-1}}}$.

\subsection{Gradient of $\mathrm{NAC}_{\bullet}$}
\label{sec:appendix:gradient-derivatives:gradient-nac-mul}

First the $\textrm{NAC}_\bullet$ is defined using scalar notation.
\begin{equation}
z_{h_\ell} = \exp\left(\sum_{h_{\ell-1}=1}^{H_{\ell-1}} W_{h_{\ell}, h_{\ell-1}} \log(|z_{h_{\ell-1}}| + \epsilon) \right)
\end{equation}

The gradient of the loss with respect to $W_{h_\ell, h_{\ell-1}}$ is straight forward to derive.
\begin{equation}
\begin{aligned}
\frac{\partial z_{h_\ell}}{\partial W_{h_\ell, h_{\ell-1}}} &= \exp\left(\sum_{h'_{\ell-1}=1}^{H_{\ell-1}} W_{h_{\ell}, h'_{\ell-1}} \log(|z_{h'_{\ell-1}}| + \epsilon) \right) \log(|z_{h_{\ell-1}}| + \epsilon) \\
&= z_{h_\ell} \log(|z_{h_{\ell-1}}| + \epsilon)
\end{aligned}
\end{equation}

We now wish to derive the backpropergation term $\delta_{h_\ell} = \frac{\partial \mathcal{L}}{\partial z_{h_\ell}}$, because $z_{h_\ell}$ affects $\{z_{h_{\ell+1}}\}_{h_{\ell+1}=1}^{H_{\ell+1}}$ this becomes:
\begin{equation}
\delta_{h_\ell} = \frac{\partial \mathcal{L}}{\partial z_{h_\ell}} = \sum_{h_{\ell+1}=1}^{H_{\ell+1}} \frac{\partial \mathcal{L}}{\partial z_{h_{\ell+1}}} \frac{\partial z_{h_{\ell+1}}}{\partial z_{h_\ell}} = \sum_{h_{\ell+1}=1}^{H_{\ell+1}} \delta_{h_{\ell+1}} \frac{\partial z_{h_{\ell+1}}}{\partial z_{h_\ell}}
\end{equation}

To make it easier to derive $\frac{\partial z_{h_{\ell+1}}}{\partial z_{h_\ell}}$ we re-express the $z_{h_\ell}$ as $z_{h_{\ell+1}}$.
\begin{equation}
z_{h_{\ell+1}} = \exp\left(\sum_{h_{\ell}=1}^{H_{\ell}} W_{h_{\ell+1}, h_{\ell}} \log(|z_{h_{\ell}}| + \epsilon) \right)
\end{equation}

The gradient of $\frac{\partial z_{h_{\ell+1}}}{\partial z_{h_\ell}}$ is then:
\begin{equation}
\begin{aligned}
\frac{\partial z_{h_{\ell+1}}}{\partial z_{h_\ell}} &= \exp\left(\sum_{h_{\ell}=1}^{H_{\ell}} W_{h_{\ell+1}, h_{\ell}} \log(|z_{h_{\ell}}| + \epsilon) \right) W_{h_{\ell+1}, h_{\ell}} \frac{\partial \log(|z_{h_{\ell}}| + \epsilon)}{\partial z_{h_\ell}} \\
&= \exp\left(\sum_{h_{\ell}=1}^{H_{\ell}} W_{h_{\ell+1}, h_{\ell}} \log(|z_{h_{\ell}}| + \epsilon) \right) W_{h_{\ell+1}, h_{\ell}} \frac{\mathrm{abs}'(z_{h_{\ell}})}{|z_{h_{\ell}}| + \epsilon} \\
&= m_{h_{\ell+1}} W_{h_{\ell+1}, h_{\ell}} \frac{\mathrm{abs}'(z_{h_{\ell}})}{|z_{h_{\ell}}| + \epsilon} 
\end{aligned}
\end{equation}

$\mathrm{abs}'(z_{h_{\ell}})$ is the gradient of the absolute function. In the paper we denote this as $\mathrm{sign}(z_{h_{\ell}})$ for brevity. However, depending on the exact defintion used there may be a difference for $z_{h_{\ell}} = 0$, as $\mathrm{abs}'(0)$ is undefined. In practicality this doesn't matter much though, although theoretically it does mean that the expectation of this is theoretically undefined when $E[z_{h_{\ell}}] = 0$.

\subsection{Gradient of NMU}
\label{sec:appendix:gradient-derivatives:gradient-nmu}

In scalar notation the NMU is defined as:
\begin{equation}
z_{h_\ell} = \prod_{h_{\ell-1}=1}^{H_{\ell-1}} \left(W_{h_{\ell-1},h_\ell} z_{h_{\ell-1}} + 1 - W_{h_{\ell-1},h_\ell} \right)
\end{equation}

The gradient of the loss with respect to $W_{h_{\ell-1},h_\ell}$ is fairly trivial. Note that every term but the one for $h_{\ell-1}$, is just a constant with respect to $W_{h_{\ell-1},h_\ell}$. The product, expect the term for $h_{\ell-1}$ can be expressed as $\frac{z_{h_\ell}}{W_{h_{\ell-1},h_\ell} z_{h_{\ell-1}} + 1 - W_{h_{\ell-1},h_\ell}}$. Using this fact, it becomes trivial to derive the gradient as:

\begin{equation}
\frac{\partial \mathcal{L}}{\partial w_{h_{\ell}, h_{\ell - 1}}} = \frac{\partial \mathcal{L}}{\partial z_{h_\ell}} \frac{\partial z_{h_\ell}}{\partial w_{h_{\ell}, h_{\ell - 1}}} = \frac{\partial \mathcal{L}}{\partial z_{h_\ell}} \frac{z_{h_\ell}}{W_{h_{\ell-1},h_\ell} z_{h_{\ell-1}} + 1 - W_{h_{\ell-1},h_\ell}} \left(z_{h_{\ell-1}} - 1\right)
\end{equation}

Similarly, the gradient $\frac{\partial \mathcal{L}}{\partial z_{h_\ell}}$ which is essential in backpropergation can equally easily be derived as:

\begin{equation}
\frac{\partial \mathcal{L}}{\partial z_{h_{\ell-1}}} = \sum_{h_\ell = 1}^{H_\ell} \frac{\partial \mathcal{L}}{\partial z_{h_\ell}} \frac{\partial z_{h_\ell}}{\partial z_{h_{\ell-1}}} = \sum_{h_\ell = 1}^{H_\ell} \frac{z_{h_\ell}}{W_{h_{\ell-1},h_\ell} z_{h_{\ell-1}} + 1 - W_{h_{\ell-1},h_\ell}} W_{h_{\ell-1},h_\ell}
\end{equation}
